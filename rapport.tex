%!TEX encoding = IsoLatin
\documentclass[12pt,ULlof,ULlot]{ULrapport}

% Chargement des packages supplementaires (si absent de la classe)
\usepackage[ansinew]{inputenc}
\usepackage[autolanguage]{numprint}
\usepackage{icomma}
%\usepackage[options]{nom_du_package}

\newcolumntype{Z}[3]{>{#1\hspace{0pt}\arraybackslash}#2{#3}}

% Definitions des parametres de la page titre
\TitreProjet{Projet de design I, titre tr�s original}
\TitreRapport{Rapport de projet -- version 0}
\Destinataire{Robert Bergevin, Luc Lamontagne et Simon Thibault}
\NumeroEquipe{7}
\NomEquipe{Nom d'�quipe}
\TableauMembres{%
   111\,111\,111  & Lauriane Blouin    & \\\hline
   222\,222\,222  & Pascal De la Rue         & \\\hline
   333\,333\,333  & Nasym Dous          & \\\hline
   444\,444\,444  & Adam El Malhouf       & \\\hline
   555\,555\,555  & Nicolas Maltais    & \\\hline
   666\,666\,666  & F�lix Pelletier   & \\\hline
   777\,777\,777  & William Poirier       & \\\hline
}
\DateRemise{22 avril 2020}


% Contenu de l'historique des versions
\HistoriqueVersions{%                        % version & date & description \\\hline
         & 17 janvier 2006 & cr�ation du document \\\hline
   1.0   & 22 janvier 2006 & illustration des capacit�s de \LaTeX: chapitre et sections, note en bas de page, r�f�rence dynamique, figure, insertion d'image, tableau, �quations, r�f�rences bibliographiques, liste\\\hline
   1.1   & 14 janvier 2007 & refonte compl�te: introduction de la classe, page titre, exemples \LaTeX, unit�s SI et r�gles d'�criture, recommandations typographiques\\\hline
   1.2   & 6 d�cembre 2007 & r�organisation des chapitres, ajout de l'historique des versions, ajouts des options ULlof et ULlot, ajout d'exemples de tableaux, corrections mineures\\\hline
   1.3   & 14 janvier 2011 & reformatage de l'exemple, changements � l'organisation des figures\\\hline
   1.3.1 & 24 novembre 2011 & matricules � 9 chiffres, titre du rapport\\\hline
   1.4   & 11 janvier 2016 & mise � jour pour la session d'hiver 2016\\\hline
	 1.4.1 &  4 janvier 2017 & mise � jour pour la session d'hiver 2017\\\hline
}


% Corps du document

\begin{document}

%   Chapitres
%!TEX encoding = IsoLatin

%
% Chapitre "Introduction"
%

\chapter{Introduction}
\label{s:intro}

Ce document se veut un exemple du format de rapport impos� pour le cours GEL--1001 Design I (m�thodologie). Son r�le est double. Il a comme premier objectif de pr�senter un exemple de rapport typique, tel que produit par \LaTeX\ en utilisant la classe \texttt{ULrapport}. Il constitue de ce fait une br�ve introduction � \LaTeX\ et � son usage afin de vous permettre de produire un rapport de qualit� professionnelle. Deuxi�mement son contenu est truff� d'informations essentielles � la r�daction d'un rapport technique de qualit�.

Un court chapitre pr�sente dans ses grandes lignes la structure d'un rapport technique.

Un bref chapitre est subs�quemment consacr� � la pr�sentation de quelques r�gles typographiques, en faisant ressortir les d�tails propres � chaque langue (fran�ais et anglais). Les �l�ments pris en compte automatiquement par le package \texttt{babel} y sont identifi�s.

Un chapitre complet est ensuite d�di� � la pr�sentation de multiples exemples d�montrant les capacit�s de \LaTeX, particuli�rement lors de l'inclusion d'objets flottants (figures ou tableaux).

Le chapitre final d�crit les notions d'�criture qui doivent �tre respect�es dans un document technique afin d'�tre conforme au Syst�me international d'unit�s~\cite{BIP98}. Y sont aussi fournies les r�gles d'�critures des nombres (valeurs et incertitudes).





%!TEX encoding = IsoLatin

\chapter{Description}
\label{s:description}

� faire



%!TEX encoding = IsoLatin

\chapter{Besoins et objectifs}
\label{s:besoins_objectifs}

� faire



%!TEX encoding = IsoLatin

\chapter{Cahier des charges}
\label{s:cahier_charges}

� faire



%!TEX encoding = IsoLatin

\chapter{Conceptualisation et analyse de faisabilit�}
\label{s:conceptualisation}

� faire



%!TEX encoding = IsoLatin

\chapter{�tude pr�liminaire}
\label{s:etude_preliminaire}

� faire



%!TEX encoding = IsoLatin

\chapter{Concept retenu}
\label{s:concept_retenu}

� faire




%!TEX encoding = IsoLatin

%
% Chapitre "Bibliographie"
%

\begin{thebibliographyUL}{99} % remplacer le "{9}" par "{99}" lorsque le nombre de references
                              % requiert 2 caracteres (>= 10 references)

\bibitem{BIP98} Organisation intergouvernementale de la Convention du M�tre, \emph{Le Syst�me international d'unit�s (SI)}, 7\ieme{} �dition, Bureau international des poids et mesures, 1998. ISBN 92--822--2154--7. R�f�rence accessible sur le site du BIPM: \url{http://www.bipm.org/fr/si/}.

\bibitem{PAR91} Roger C. \bsc{Parker} et Lise \bsc{Th�rien}, \emph{Mise en page --- Un guide de conception graphique sur micro-ordinateur}, Les �ditions Reynald Goulet Inc., 1991.
ISBN 2--8937--7045--2.

\bibitem{KNU84} Donald E. \bsc{Knuth}, \emph{The \TeX book}, Addison Wesley Professional, 1984. ISBN 0--201--13448--9.

\bibitem{LAM94} Leslie \bsc{Lamport}, \emph{\LaTeX: A document Preparation System}, 2\ieme{} �dition, Addison Wesley Professional, 1994. ISBN 0--201--52983--1.

\bibitem{MIT04} Frank \bsc{Mittelbach}, Michel \bsc{Goossens}, Johannes \bsc{Braams}, David \bsc{Carlisle} et Chris \bsc{Rowley}, \emph{The \LaTeX{} Companion}, 2\ieme{} �dition, Addison Wesley, 2004. ISBN 0--201--36299--6.

\bibitem{BIB07} Biblioth�que de l'Universit� Laval. \emph{Comment citer un document �lectronique?}, [En ligne]. \url{http://www.bibl.ulaval.ca/doelec/doelec29.html} (Page consult�e le 14 janvier 2007)

\bibitem{BIP00} Organisation intergouvernementale de la Convention du M�tre, \emph{Le Syst�me international d'unit�s (SI) --- Suppl�ment 2000: additions et corrections � la 7\ieme~�dition (1998)}, Bureau international des poids et mesures, 2000. R�f�rence accessible sur le site du BIPM: \url{http://www.bipm.org/fr/si/}.

\bibitem{TAY95} Barry N. \bsc{Taylor}, \emph{Guide for the Use of the International System of Units (SI)}, NIST Special Publication 811, 1995. R�f�rence accessible sur le site du NIST: \url{http://physics.nist.gov/Pubs/pdf.html}.

\bibitem{KLA96} Klaas B. \bsc{Klaassen}, \emph{Electronic Measurement and Instrumentation}, Cambridge University Press, 1996. ISBN 0--521--47729--8.

\bibitem{MOH02} Peter J. \bsc{Mohr} et Barry N. \bsc{Taylor}, \og CODATA Recommended Values of the Fundamental Physical Constants: 2002 \fg, \emph{Review of Modern Physics}, \textbf{77}, p.~1--107 (2005).
\end{thebibliographyUL}








%   Annexes
\appendix
\input{tex/liste_sig_acr}

\end{document}
% Fin du document

