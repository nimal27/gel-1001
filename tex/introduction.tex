%!TEX encoding = IsoLatin

%
% Chapitre "Introduction"
%

\chapter{Introduction}
\label{s:intro}

Ce document se veut un exemple du format de rapport impos� pour le cours GEL--1001 Design I (m�thodologie). Son r�le est double. Il a comme premier objectif de pr�senter un exemple de rapport typique, tel que produit par \LaTeX\ en utilisant la classe \texttt{ULrapport}. Il constitue de ce fait une br�ve introduction � \LaTeX\ et � son usage afin de vous permettre de produire un rapport de qualit� professionnelle. Deuxi�mement son contenu est truff� d'informations essentielles � la r�daction d'un rapport technique de qualit�.

Un court chapitre pr�sente dans ses grandes lignes la structure d'un rapport technique.

Un bref chapitre est subs�quemment consacr� � la pr�sentation de quelques r�gles typographiques, en faisant ressortir les d�tails propres � chaque langue (fran�ais et anglais). Les �l�ments pris en compte automatiquement par le package \texttt{babel} y sont identifi�s.

Un chapitre complet est ensuite d�di� � la pr�sentation de multiples exemples d�montrant les capacit�s de \LaTeX, particuli�rement lors de l'inclusion d'objets flottants (figures ou tableaux).

Le chapitre final d�crit les notions d'�criture qui doivent �tre respect�es dans un document technique afin d'�tre conforme au Syst�me international d'unit�s~\cite{BIP98}. Y sont aussi fournies les r�gles d'�critures des nombres (valeurs et incertitudes).




