%!TEX encoding = IsoLatin

%
% Chapitre "Recommandations typographiques"
%

\chapter{Recommandations typographiques}
\label{s:typo}

Ce bref chapitre d�crit quelques r�gles typographiques de base applicables aux textes r�dig�s en fran�ais ou en anglais. Elles sont d�crites aux tableaux \ref{t:recommandations_typographiques} et \ref{t:recommandations_typographiques_legende}.
\begin{table}[htp]
   \footnotesize
   \centering
   \caption{Recommandations typographiques propres aux langues fran�aise et anglaise. L'ast�risque d�note celles qui sont prises en compte automatiquement par le package \texttt{babel}.}
   \label{t:recommandations_typographiques}
   \begin{tabular}{|l|l|c|c|l|c|c|}
      \hline\hline
      \multicolumn{1}{|c|}{}
      & \multicolumn{3}{c|}{\emph{fran�ais}}
      & \multicolumn{3}{c|}{\emph{anglais}}
      \\\cline{2-7}
      \multicolumn{1}{|c|}{\raisebox{1.4ex}[0ex][0ex]{\emph{�l�ment}}}
         % la commande "\raisebox" peut etre evitee en utilisant le package "multirow"
      & \multicolumn{1}{c|}{\emph{nom}}
      & \emph{signe}
      & \emph{code}
      & \multicolumn{1}{c|}{\emph{nom}}
      & \emph{signe}
      & \emph{code}
      \\\hline

      guillemets$^\ast$
      & doubles chevrons & \og\ldots\fg & \verb|\og \fg|
      & double quotes    & ``\ldots''   & \verb|`` ''|
      \\

      liste � puces$^\ast$
      & tiret moyen      & --           & \verb|--|
      & bullet           & $\bullet$    & \verb|$\bullet$|
      \\

      abr�viation
      & c'est-�-dire     & c.-�-d.      &
      & \emph{id est}    & \emph{i.e.}  &
      \\

      abr�viation
      & par exemple              & p.\ ex.      &
      & \emph{exempli gracia}    & \emph{e.g.}  &
      \\

      abr�viation
      & pages            & p.      &
      & pages            & pp.     &
      \\

      \hline\hline
   \end{tabular}
\end{table}

\begin{table}[htp]
   \footnotesize
   \centering
   \caption{Recommandations propres aux langues fran�aise et anglaise quant � l'emplacement des l�gendes. Notez que ces r�gles ne sont pas prises en compte par le package \texttt{babel}.}
   \label{t:recommandations_typographiques_legende}
   \begin{tabular}{|l|l|l|}
      \hline\hline
      \multicolumn{1}{|c|}{\emph{objet flottant}}
      & \multicolumn{1}{c|}{\emph{fran�ais}}
      & \multicolumn{1}{c|}{\emph{anglais}}
      \\\hline

      figure
      & apr�s l'objet graphique
      & apr�s l'objet graphique
      \\

      table
      & avant la grille
      & apr�s la grille
      \\

      \hline\hline
   \end{tabular}
\end{table}

Toute figure et tout tableau doit �tre r�f�renc� dans le texte du rapport par son num�ro (p. ex. \og figure~\ref{f:exemple_figure_pdf} \fg\  ou \og tableau~\ref{t:prix_materiaux} \fg). Dans un rapport r�dig� en anglais, contrairement � un rapport en fran�ais, la majuscule est toujours utilis�e (p. ex. \og Figure~\ref{f:exemple_figure_pdf} \fg\  ou \og Table~\ref{t:prix_materiaux} \fg).
